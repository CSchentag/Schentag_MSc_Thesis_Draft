\chapter{Introduction}

\section{Overview}
Since the first satellites were launched, scientists have been designing instrumentation for these satellites to learn as much as possible about the space and atmospheric environment. An area that has become popular in recent decades is atmospheric remote sensing from space. These instruments use various interactions of radiation with the atmosphere to detect and measure levels of gases and aerosols in the atmosphere. These measurements are often classified as either limb-viewing or nadir-viewing. Nadir-viewing instruments have a downward view, towards the Earth, while limb-viewing instruments look through the atmosphere, tangent to Earth. Limb-viewing instruments tend to focus on stratospheric measurements as the techniques allows for the measurement of the vertical distribution of temperature, pressure, and various species of aerosols and gases. These instruments are becoming more popular due to their increased global coverage and high vertical resolution ~\citep{SPARC}.

\section{LIFE}
This thesis discusses the Limb Imaging Fourier transform spectrometer Experiment (LIFE) instrument, under development at the University of Saskatchewan. It is a thermal emission imaging instrument designed for atmospheric remote sensing of greenhouse gases from a high-altitude balloon and is built in collaboration with the Canadian Space Agency (CSA) and ABB. The long-term goal of the LIFE instrument is to be part of a satellite mission in low earth orbit. LIFE is a group project consisting of multiple graduate students at the University of Saskatchewan. Specifically, this report focuses on my contribution, two aspects of the instrument that forms the core of the M.Sc. thesis work: the thermal-mechanical design, and the Mercury Cadmium Telluride (MCT) infrared detector characterization. 

\section{Outline}
This thesis discusses the thermal-mechanical design of the first balloon-borne prototype of the LIFE instrument, as well as the characterization of the MCT infrared detector. Chapter \ref{bkgnd} presents background for the rest of the thesis. Background is given for limb imagining in the upper troposphere/lower stratosphere region (UTLS), and previous instruments that are the inspiration for the LIFE instrument. Background on the thermal regime of the balloon flight and environment is also discussed, including thermal phenomena, thermal controls and the thermal designs of similar instruments. Finally, this chapter also contains background on different types of infrared detectors, why the MCT Detector was chosen, and issues to be characterized.

One of the two main aspects of this thesis is the thermal-mechanical design, which is discussed in Chapter \ref{thermal}. This chapter discusses the requirements for the thermal design, both from the optical system and the electronics. It also discusses the mechanical requirements for the instrument and the flight on-board the CNES gondola. The thermal environment is described in more detail, and the operations of the software used for the simulations is described. The majority of this chapter is the process of the design of the thermal and mechanical design of the LIFE instrument, through a variety of iterations, simulations and environments.

Following the flight of the LIFE instrument in the late summer, the thermal model was compared to temperatures seen in flight. This is discussed in Chapter \ref{postflight}. It also discusses the building of a full flight model for future atmospheric flight instruments for all stages of the flight. This section also covers the results of the mechanical design and the result of the flight.

The IR detector in the LIFE instrument needed to be characterized with settings chosen for optimal measurements. The process of this characterization is described in Chapter \ref{detector}. The detector itself is described in detail, as well as the process and results. Finally, the noise of the detector as it relates to its thermal properties are discussed for flight, as well as the performance of the detector in general during flight.

Chapter \ref{future} goes into detail on the future work necessary for LIFE and the thermal model of the instrument. Thermal-mechanical changes based on what was seen during flight are discussed, as well as recommended updates to the atmospheric instrument thermal model.