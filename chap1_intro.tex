\chapter{Introduction}

\section{Overview}
Since the first satellites were launched, scientists have been designing instrumentation for these satellites to learn as much as possible about the space and atmospheric environment. An area that has become popular in recent decades is atmospheric remote sensing from space. These instruments use various interactions of radiation with the atmosphere to detect and measure levels of gases and aerosols in the atmosphere. These measurements are often classified as either limb-viewing or nadir-viewing. Nadir-viewing instruments have a downward view, towards the Earth, while limb-viewing instruments look through the atmosphere, tangent to Earth. Limb-viewing instruments tend to focus on stratospheric measurements as the techniques allows for the measurement of the vertical distribution of temperature, pressure, and various species of aerosols and gases. These instruments are becoming more popular due to their increased global coverage and high vertical resolution ~\citep{SPARC}.

\section{LIFE}
The Limb Imaging Fourier Transform Experiment (LIFE) instrument is a remote sensing atmospheric instrument under development at the University of Saskatchewan. It is a thermal emission imaging instrument designed for the measurement of greenhouse gases from a high-altitude balloon platform, and is built in collaboration with the Canadian Space Agency (CSA) and ABB. The long-term goal of the LIFE instrument is to be part of a satellite mission in low earth orbit. LIFE is a group project consisting of multiple graduate students at the University of Saskatchewan, as part of the Atmospheric Research Group in ISAS. It is funded through the Flights and Fieldwork for the Advancement of Science and Technology (FAST) project, from the CSA. 

LIFE is designed to use new imaging Fourier Transform Spectrometer (IFTS) technology to image the atmosphere in the thermal regime. It is the second atmospheric instrument to be developed around an IFTS, following the successful development and operation of the GLORIA instrument from the Karlsruhe Institute of Technology (KIT). The improvement in technology of GLORIA of LIFE over previous atmospheric thermal imaging instruments, such as MIPAS, is the vertical imaging capabilities with the IFTS. Previous FTS based instruments took atmospheric images through the use of a single pixel scanning the atmosphere; The use of a two dimensional pixel array allows single images to be taken and avoid the need for scanning~\citep{GLORIA_objectives}. LIFE is designed in the footsteps of GLORIA, which aims to meet the capabilities of GLORIA while creating an instrument that is less expensive and has a smaller footprint. 

The instrument is developed to measure trace greenhouse gases in the UTLS region. These greenhouse gases play a critical role in climate change, and it is important that information is gathered to inform climate change models. There is a gap in knowledge of key greenhouse gases in this region of the atmosphere, and the LIFE and GLORIA instruments aim to close that gap and provide measurements on levels of various constituents in this region. Some of the important greenhouse gases that are measured by the LIFE instrument are as follows: H\textsubscript{2}O, O\textsubscript{3}, N\textsubscript{2}O, and CH\textsubscript{4}. Operating in a similar spectral range to GLORIA, it is designed to measure in the wavenumber region of 700 cm\textsuperscript{-1} to 1400 cm\textsuperscript{-1}. It will take these measurements from the lower stratosphere, at an altitude of 35 km. The instrument will image vertically from this altitude down to an altitude of 8km~\citep{ethans_thesis}. This contrasts to the GLORIA instrument, which took measurements from an aircraft at lower altitudes.

The first version of the instrument, as developed in this thesis, is a prototype designed to fly on a high-altitude balloon at the aforementioned altitude of 35 km. As a prototype it is designed to demonstrate that a IFTS based thermal imaging instrument can be developed and take good measurements for a reasonable cost and size. It will inform future designs of the instrument, eventually leading to a satellite based design. The initial development of the LIFE instrument, including the core optical design and the initial modelling of the optical system, was done by Ethan Runge for his MSc. thesis. This thesis discusses two core tasks of the development of this core prototype: The thermal-mechanical design, and the characterization of the infrared detector.

\section{Outline}
This thesis discusses the thermal-mechanical design of the first balloon-borne prototype of the LIFE instrument, as well as the characterization of the MCT infrared detector. Chapter \ref{bkgnd} presents background for the rest of the thesis. Background is given for limb imagining in the upper troposphere/lower stratosphere region (UTLS), and previous instruments that are the inspiration for the LIFE instrument. Background on the thermal regime of the balloon flight and environment is also discussed, including thermal phenomena, thermal controls and the thermal designs of similar instruments. Finally, this chapter also contains background on different types of infrared detectors, why the MCT Detector was chosen, and issues to be characterized.

One of the two main aspects of this thesis is the thermal-mechanical design, which is discussed in Chapter \ref{thermal}. This chapter discusses the requirements for the thermal design, both from the optical system and the electronics. It also discusses the mechanical requirements for the instrument and the flight on-board the CNES gondola. The thermal environment is described in more detail, and the operations of the software used for the simulations is described. The majority of this chapter is the process of the design of the thermal and mechanical design of the LIFE instrument, through a variety of iterations, simulations and environments.

Following the flight of the LIFE instrument in the late summer, the thermal model was compared to temperatures seen in flight. This is discussed in Chapter \ref{postflight}. It also discusses the building of a full flight model for future atmospheric flight instruments for all stages of the flight. This section also covers the results of the mechanical design and the result of the flight.

The IR detector in the LIFE instrument needed to be characterized with settings chosen for optimal measurements. The process of this characterization is described in Chapter \ref{detector}. The detector itself is described in detail, as well as the process and results. Finally, the noise of the detector as it relates to its thermal properties are discussed for flight, as well as the performance of the detector in general during flight.

Chapter \ref{future} goes into detail on the future work necessary for LIFE and the thermal model of the instrument. Thermal-mechanical changes based on what was seen during flight are discussed, as well as recommended updates to the atmospheric instrument thermal model.